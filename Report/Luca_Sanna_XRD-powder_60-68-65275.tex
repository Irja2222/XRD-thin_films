\documentclass[a4paper,twocolumn,12 pt]{article}
\usepackage{silence}  % pacchetto che permette di sopprimere warnings cagacazzo
\WarningFilter{latex}{`h' float specifier changed to `ht'}
\usepackage{geometry}
 \geometry{
 a4paper,
 total={170mm,257mm},
 left=20mm,
 top=20mm,
 }
\usepackage[T1]{fontenc}
\usepackage[utf8]{inputenc}
\usepackage[english,italian]{babel}
\usepackage{url}
\usepackage{graphicx}
\usepackage{amsmath}
\usepackage{amstext}
\usepackage{booktabs}
\usepackage{wrapfig}
\usepackage{siunitx}
\usepackage{subcaption}
\usepackage{amssymb}
\usepackage{animate}
\usepackage{csquotes}
\usepackage{biblatex}
\usepackage{xurl}   % compatta gli url
\usepackage{hyperref}
\usepackage{indentfirst}
\usepackage[font=footnotesize, labelfont=bf]{caption}
% \usepackage[LGRgreek]{mathastext}
\usepackage{esint}
\makeatletter
% make esint definition in line with amsmath
\@for\next:={int,iint,iiint,iiiint,dotsint,oint,oiint,sqint,sqiint,
  ointctrclockwise,ointclockwise,varointclockwise,varointctrclockwise,
  fint,varoiint,landupint,landdownint}\do{%
    \expandafter\edef\csname\next\endcsname{%
      \noexpand\DOTSI
      \expandafter\noexpand\csname\next op\endcsname
      \noexpand\ilimits@
    }%
  }
\makeatother
\usepackage{multirow}
\usepackage{multicol}
\usepackage{float}
\usepackage{subdepth}

\addbibresource{citations.bib}
\hbadness=1000000

\pagestyle{plain}

\title{\textbf{Studio dell'orientazione di cristalli singoli in zaffiro}}
\author{\textbf{Luca Sanna, 60/68/65275}}
\date{\textbf{10/12/2025}}


\begin{document}
\twocolumn[
\maketitle
  \begin{@twocolumnfalse}\maketitle
        \begin{abstract}\noindent
          L'obiettivo dell'esperienza di laboratorio è studiare il diffrattogramma XRD, ottenuto secondo geometria Bragg-Brentano, di sei diversi film sottili in allumina-$\alpha$ (Al$_2$O$_3$), comunemente chiamata zaffiro, in fase di cristallo singolo, con il fine di stimare le rispettive orientazioni privilegiate. Di nota, si è identificato, per il campione di denominazione III, l'orientazione M, di indici di Miller $\left( 3 \ 0 \ 0 \right)$, ad un angolo di diffrazione di $\left( 68.23 \pm 0.07 \right)~deg$, che, entro l'incertezza, è confrontabile con le stime osservate in letteratura.
        \end{abstract}
  \end{@twocolumnfalse}
]

\raggedbottom
\section*{Introduzione}
La tecnica di indagine XRD (X-Rays Diffraction) fa uso dell'interazione radiazione-materia che scaturisce tra luce incidente X e un campione in fase cristallina. Poiché le tipiche distanze interatomiche per un cristallo sono comparabili con le lunghezze d'onda $\lambda$ della radiazione X ($\approx 1-10~\text{\AA}$), vi sono le condizioni perché il fenomeno della diffrazione avvenga.
\begin{figure}[htbp]
  \centering
  \includegraphics[width = 1\linewidth]{../von_Laue.png}
  \caption{Schema esemplificativo della legge di von Laue.}
  \label{fig: von Laue}
\end{figure}

Usando la figura \ref{fig: von Laue} come riferimento, il fascio luminoso segue la direzione di incidenza, definita dal vettore d'onda $\vec{k}_{in} = 2\pi/\lambda$ e dall'angolo di incidenza $\vartheta$, e interagisce con i piani, costituenti il campione. Il fascio risultante segue dunque la direzione di diffrazione, definita dal vettore d'onda $\vec{k}_{out}$ e dall'angolo $2\vartheta$. La differenza di cammino ottico tra i due percorsi che compie la luce, interagendo tra due generici atomi del reticolo, la cui distanza è definita dal vettore reticolare $\vec{R}$, interferiscono costruttivamente se è rispettata la seguente condizione:
\begin{equation}
  \left(\vec{k}_{out} - \vec{k}_{in}\right) \circ \vec{R} = 2 \pi m
\end{equation}
dove $m \in \mathbb{N}$ è l'ordine diffrattivo (si considera il prim'ordine perché il più intenso).


\section*{Esperimento}


\section*{Risultati}



\section*{Conclusioni}



\onecolumn
\newpage
% \printbibliography


\end{document}