\documentclass[a4paper,twocolumn,12 pt]{article}
\usepackage{silence}  % pacchetto che permette di sopprimere warnings cagacazzo
\WarningFilter{latex}{`h' float specifier changed to `ht'}
\usepackage{geometry}
 \geometry{
 a4paper,
 total={170mm,257mm},
 left=20mm,
 top=20mm,
 }
\usepackage[T1]{fontenc}
\usepackage[utf8]{inputenc}
\usepackage[english,italian]{babel}
\usepackage{url}
\usepackage{graphicx}
\usepackage{amsmath}
\usepackage{amstext}
\usepackage{booktabs}
\usepackage{wrapfig}
\usepackage{siunitx}
\usepackage{subcaption}
\usepackage{amssymb}
\usepackage{animate}
\usepackage{csquotes}
\usepackage{biblatex}
\usepackage{xurl}   % compatta gli url
\usepackage{hyperref}
\usepackage{indentfirst}
\usepackage[font=footnotesize, labelfont=bf]{caption}
% \usepackage[LGRgreek]{mathastext}
\usepackage{esint}
\makeatletter
% make esint definition in line with amsmath
\@for\next:={int,iint,iiint,iiiint,dotsint,oint,oiint,sqint,sqiint,
  ointctrclockwise,ointclockwise,varointclockwise,varointctrclockwise,
  fint,varoiint,landupint,landdownint}\do{%
    \expandafter\edef\csname\next\endcsname{%
      \noexpand\DOTSI
      \expandafter\noexpand\csname\next op\endcsname
      \noexpand\ilimits@
    }%
  }
\makeatother
\usepackage{multirow}
\usepackage{multicol}
\usepackage{float}
\usepackage{subdepth}

\addbibresource{citations.bib}
\hbadness=1000000000

\pagestyle{plain}

\title{\textbf{Studio dell'orientazione di film sottili, in fase di cristallo singolo, in zaffiro}}
\author{\textbf{Luca Sanna, 60/68/65275}}
\date{\textbf{22/12/2025}}


\begin{document}
\twocolumn[
\maketitle
  \begin{@twocolumnfalse}\maketitle
        \begin{abstract}\noindent
          L'obiettivo dell'esperienza di laboratorio è studiare il diffrattogramma a raggi X, ottenuto secondo geometria Bragg-Brentano, di sei diversi film sottili in allumina-$\alpha$ (Al$_2$O$_3$, comunemente chiamata zaffiro), in fase di cristallo singolo, con il fine di stimare le rispettive orientazioni privilegiate. Di nota, si è identificato, per il campione di denominazione III, l'orientazione M, di indici di Miller $\left( 3 \ 0 \ 0 \right)$, ad un angolo di diffrazione di $\left( 68.23 \pm 0.07 \right)~deg$, che, entro l'incertezza, è confrontabile con le stime osservate in letteratura.
        \end{abstract}
  \end{@twocolumnfalse}
]

\raggedbottom
\section*{Introduzione}
La tecnica di indagine XRD (X-Rays Diffraction) fa uso dell'interazione radiazione-materia che scaturisce tra luce incidente X e un campione in fase cristallina. Poiché le tipiche distanze interatomiche per un cristallo sono comparabili con le lunghezze d'onda $\lambda$ della radiazione X ($\approx 1-10~\text{\AA}$), vi sono le condizioni perché il fenomeno della diffrazione avvenga.
\begin{figure}[htbp]
  \centering
  \includegraphics[width = 1\linewidth]{../Images/von_Laue.png}
  \caption{Schema esemplificativo della legge di von Laue.}
  \label{fig: von Laue}
\end{figure}

Usando la figura \ref{fig: von Laue} come riferimento, il fascio luminoso segue la direzione di incidenza, definita dal vettore d'onda $\vec{k}_{in} = 2\pi/\lambda$ e dall'angolo di incidenza $\vartheta$, e interagisce con i piani costituenti il campione. Il fascio risultante segue dunque la direzione di diffrazione, definita dal vettore d'onda $\vec{k}_{out}$ e dall'angolo $2\vartheta$. La differenza di cammino ottico tra i due percorsi che compie la luce interagendo tra due generici atomi del reticolo, la cui distanza è definita dal vettore reticolare $\vec{R}$, interferiscono costruttivamente se è rispettata la seguente condizione:
\begin{equation}
  \left(\vec{k}_{out} - \vec{k}_{in}\right) \circ \vec{R} = 2 \pi m
  \label{eq: von Laue}
\end{equation}
dove $m \in \mathbb{N}$ è l'ordine diffrattivo (si considera il prim'ordine perché il più intenso).

Il risultato della misura è un pattern d'intensità dei fotoni diffratti punto per punto sullo schermo di acquisizione. Gradi di simmetria differenti del cristallo corrispondono sistematicamente a pattern di diffrazione differenti. È intuitivo dunque modellizzare l'ambiente di predizione secondo un nuovo reticolo, sotto il nome di \emph{reticolo reciproco}, che schematizza, analogamente al reticolo diretto per i siti reticolari, i centri diffrattivi; anziché parlare di vettore reticolare, si fa riferimento a un \emph{vettore reticolare reciproco}, $G = 2\pi/d_{hkl}$, dove $d_{hkl}$ è la distanza interplanare che caratterizza la famiglia di piani, individuata dalla controparte reciproca degli indici planari diretti, gli \emph{indici di Miller}, $(h \ k \ l)$. Poiché questi numeri possono essere anche negativi, è convenzione scrivere, per un certo indice $j$, l'equivalenza $-j \equiv \overline{j}$.

L'equazione \ref{eq: von Laue} assume così la forma:
\begin{equation}
  \vec{G} \circ \vec{R} = 2\pi m,
  \label{eq: von Laue, G}
\end{equation}
e prende il nome di \emph{legge di Von Laue}.

Gli indici di Miller non sono l'unico modo di contraddistinguere una famiglia di piani. Infatti, gli \emph{indici esagonali} sono convenzionalmente utilizzati per descrivere l'orientazione di famiglie di piani facenti parte reticoli di struttura, appunto, esagonale. La legge di conversione dagli indici di Miller $(h \ k \ l)$ agli indici esagonali $(h \ k \ i \ l)$ sia:
\begin{equation}
  \begin{split}
    &h \longrightarrow h\\
    &k \longrightarrow k\\
    &i \longrightarrow -\left(h + k\right)\\
    &l \longrightarrow l
  \end{split}
  \label{eq: hex}
\end{equation}
In definitiva, costruendo un grafico delle intensità dei fotoni diffratti in funzione dell'angolo di diffrazione, curva che prende il nome di \emph{diffrattogramma}, è possibile associare ad ogni picco una specifica combinazione di indici di Miller, e così la famiglia di piani di orientazione associata.


\section*{Esperimento}
\begin{figure}[H]
  \centering
  \includegraphics[width = 1\linewidth]{../Images/sapphire-struct_upscaled_croped.png}
  \caption{Orientazioni dei piani cristallografici funzionali dello zaffiro, di gruppo spaziale "R-3c:H"~\supercite{sapphire}.}
  \label{fig: sapphire multi}
\end{figure}
I campioni osservati sono sei film sottili, in fase di singolo cristallo, di $\alpha$-allumina (Al$_2$O$_3$), materiale comunemente chiamato zaffiro, che differiscono nell'orientazione privilegiata della famiglia di piani caratterizzante; i campioni si identificano come I, II, III, IV, V e VI.
\begin{figure}[htbp]
    \centering
    \begin{minipage}{0.23\textwidth}
        \centering
        \includegraphics[width = 1\linewidth]{../Images/sapphire.png}
        \caption*{(a)}
    \end{minipage}%
    \hspace{0.01\textwidth}
    \begin{minipage}{0.23\textwidth}
        \centering
        \includegraphics[width = 1\linewidth]{../Images/Crystallographic-diagram-of-sapphire_upscaled_croped.png}
        \vspace*{3.5mm}\caption*{(b)}
    \end{minipage}
    \caption{Struttura dello zaffiro, di gruppo spaziale "R-3c:H", con particolare attenzione alle orientazioni dei piani cristallografici A, C, M, R in figura (a)~\supercite{sapphireR}, e del piano cristallografico N in figura (b)~\supercite{sapphireN}.}
    \label{fig: sapphire standing}
\end{figure}

Le figure \ref{fig: sapphire multi} e \ref{fig: sapphire standing} illustrano la struttura reticolare dello zaffiro studiato.

Il gruppo di simmetria spaziale associato sia "R-3c:H" ($167$), in cui la dicitura ":H" identifica la disposizione esagonale (\emph{Hexagonal}) degli atomi nel reticolo. In particolare, la figura \ref{fig: sapphire multi} mette in evidenza i piani A, C, M e R, mentre la figura \ref{fig: sapphire standing} rimarca questi ultimi e introduce il piano N; si noti come non vi sia una combinazione unica di indici esagonali a definire una famiglia di piani, ma lo stesso piano è individuato da diverse possibili orientazioni.
\begin{figure}[htbp]
  \centering
  \includegraphics[width = 1\linewidth]{../Images/Gobel.png}
  \caption{Schema strutturale di un XRD, secondo geometria di Bragg-Brentano e correzione della sorgente mediante specchio di G\"{o}bel.}
  \label{fig: Gobel}
\end{figure}

Inoltre, casistica non visualizzabile nelle figure appena descritte, si associano, allo stesso piano, direzioni planari definite da indici di Miller che sono multipli di una combinazione irriducibile.
\begin{figure*}[t!]
    \centering
    \includegraphics[width = 1\linewidth]{../plot_allofit_normalized.pdf}
    \caption{Diffrattogrammi a raggi X, acquisiti secondo geometria Bragg-Brentano, di film sottili di zaffiro, in fase di singolo cristallo; necessitano di particolare attenzione i campioni IV e VI che condividono la stessa orientazione N.}
    \label{fig: plots}
\end{figure*}

\noindent Esempio calzante al lavoro di laboratorio affrontato, è l'orientazione $(3 \ 0 \ 0)$, che vanta $3$ volte lo spessore lungo la direzione $h$ dell'orientazione $(1 \ 0 \ 0)$.

L'apparecchio strumentale utilizzato~\supercite{XRD:spec}, oltre a mettere a disposizione la geometria Bragg-Brentano, dispone dell'opzione di regolare la direzione del raggio incidente mediante specchi di G\"{o}bel.

Con riferimento alla figura \ref{fig: Gobel}, essi consentono di correggere la triettoria dispersiva di una sorgente X, tramite \emph{riflessione di Bragg}, e convogliarla sul campione in un flusso parallelo; è così possibile migliorare l'esame XRD di un film sottile, in fase di singolo cristallo, con i vantaggi~\supercite{gobel} di un'intensità maggiore del fascio incidente sul sito di osservazione, un rateo segnale-rumore più elevato, l'aggiustamento più semplice e rapido del campione all'interno del sistema di misura e la soppressione effettiva di radiazione estranea.

Al fine di ottimizzare il processo di parallelizzazione, si fa uso inoltre di una combinazione di \emph{fenditure di Soller} e fenditure allineatrici, sia in incidenza sia in diffrazione.


\section*{Risultati}
Il primo passo, al fine di associare a una determinata famiglia di piani un picco diffrattivo, è individuare l'angolo di diffrazione associato; l'errore sulla stima è considerato come larghezza a mezza altezza, considerando la maggiore delle due distanze in $2\vartheta$ tra massimo e metà altezze, in caso di forma asimmetrica del picco.
\begin{table}[htbp]\setlength\belowcaptionskip{-5pt}
    \centering
\begin{tabular}{cc}
    \hline
    \textbf{campione} & $\mathbf{2\vartheta}$ ($deg$)\\
    \hline
    I & $37.81 \pm 0.05$\\
    II & $41.75 \pm 0.03$\\
    III & $68.23 \pm 0.07$\\
    IV & $43.44 \pm 0.04$\\
    \multirow{2}{*}{V} & $25.65 \pm 0.06$\\
    & $52.60 \pm 0.05$\\
    VI & $43.32 \pm 0.02$\\
    \hline
\end{tabular}
\caption{Picchi di diffrazione di film sottili, in fase di singolo cristallo, di zaffiro di gruppo spaziale "R-3c:H".}
\label{tab: picchi}
\end{table}

La tabella \ref{tab: picchi} elenca le stime effettuate. L'atto successivo è confrontare i grafici rilevati con diffrattogrammi in letteratura di polveri dello stesso materiale. Infatti, per definizione, una polvere è idealmente un agglomerato di infiniti cristalliti, ed è dotata di infinite orientazioni casuali; per questo motivo, da un singolo diffrattogramma, è potenzialmente possibile ottenere un'informazione completa su tutte le direzioni planari che caratterizzano un campione. Si accede dunque a un database~\supercite{cod} di file ".cif" (Crystallographic Information File) e si ricercano le pubblicazioni che corrispondono al materiale di interesse; per l'Al$_2$O$_3$, di gruppo spaziale
"R-3c:H", si è usufruito delle misure identificate dai "COD ID" $1000017$ e $1000032$, osservate tramite il software di analisi cristallografica \emph{Mercury}~\supercite{mercury}.
\begin{table}[htbp]\setlength\belowcaptionskip{-5pt}
    \centering
\begin{tabular}{ccc}
    \hline
    \multirow{2}{*}{$\mathbf{2\vartheta}$ ($deg$)} & \textbf{indici} & \textbf{indici}\\
    & \textbf{di Miller} & \textbf{esagonali}\\
    \hline
    $37.8 \pm 0.1$ & $(2 \ \overline{1} \ 0)$ & $(2 \ \overline{1} \ \overline{1} \ 0)$\\
    $41.7 \pm 0.1$ & $(0 \ 0 \ 6)$ & $(0 \ 0 \ 0 \ 6)$\\
    $68.2 \pm 0.1$ & $(3 \ 0 \ 0)$ & $(3 \ 0 \ \overline{3} \ 0)$\\
    $43.3 \pm 0.1$ & $(2 \ \overline{1} \ 3)$ & $(2 \ \overline{1} \ \overline{1} \ 3)$\\
    $25.6 \pm 0.1$ & $(1 \ 0 \ \overline{2})$ & $(1 \ 0 \ \overline{1} \ 2)$\\
    $52.5 \pm 0.1$ & $(2 \ 0 \ \overline{4})$ & $(2 \ 0 \ \overline{2} \ 4)$\\
    $43.3 \pm 0.1$ & $(2 \ \overline{1} \ 3)$ & $(2 \ \overline{1} \ \overline{1} \ 3)$\\
    \hline
\end{tabular}
\caption{Picchi di diffrazione di polveri in zaffiro, di gruppo spaziale "R-3c:H", e orientazioni privilegiate di interesse associate.}
\label{tab: orientazioni}
\end{table}

La tabella \ref{tab: orientazioni} elenca gli esiti delle ricerche; come anticipato nella sezione \emph{Esperimento}, si consta la presenza di orientazioni che possiedono indici di Miller non irriducibili. Comparando la tabella \ref{tab: picchi}, la tabella \ref{tab: orientazioni}, e le figure \ref{fig: sapphire multi} e \ref{fig: sapphire standing}, si ottiene infine la figura \ref{fig: plots}.

Si noti in primo luogo la presenza sia dell'emissione K$\alpha$ che di quella K$\beta$; per convenzione, si è valutata soltanto la prima al fine di determinare i picchi diffrattivi.

È inoltre importante sottolineare come i massimi di intensità dei campioni IV e VI quasi si sovrappongono; per questo motivo, si è considerato sufficiente indicare la rispettiva orientazione N una sola volta.

Andando oltre con lo studio del grafico \ref{fig: plots}, i diffrattogrammi dei campione III e VI colgono l'attenzione a causa di un background significativamente rumoroso. Poiché si è scelto di normalizzare tutte le curve ai rispettivi massimi, si perde l'informazione sull'effettivo numero dei conteggi; infatti, i grafici in questione vantano picchi di rispettivamente $\approx 0.5 \ counts$ e $\approx 1.5 \ counts$, a indice della qualità non ottimale delle misure (in appendice sarà disponibile il grafico \ref{fig: plots} grezzo).

In conclusione, si distingue il campione V perché manifesta due diversi picchi diffrattivi per la stessa orientazione R; si individua la differenza nella molteplicità doppia, nelle direzioni $h$ e $l$, del piano $(2 \ 0 \ \overline{4})$ rispetto al piano $(1 \ 0 \ \overline{2})$.


\section*{Conclusioni}
L'obiettivo dell'esperienza di laboratorio è stato rilevare i diffrattogrammi a raggi X di sei film sottili di zaffiro, in fase di singolo cristallo, diversi per orientazione privilegiata. Successivamente, si è posto il problema di riconoscere, nei picchi diffrattivi ottenuti, le direzioni planari tipiche del gruppo spaziale del campione, "R-3c:H", tramite confronto con i dati in letteratura.

La comparazione tra la tabella \ref{tab: picchi} e la tabella \ref{tab: orientazioni} permette di affermare che i valori osservati sperimentalmente e ricercati in letteratura sono confrontabili entro i rispettivi errori di misura; inoltre, errori relativi massimi di rispettivamente $0.39 \%$ e $0.24 \%$ sostengono la validità di questa corrispondenza. Essa ha ulteriormente concesso di poter individuare con precisione sufficiente, seguendo gli schemi dettati dalle figure \ref{fig: sapphire multi} e \ref{fig: sapphire standing}, le famiglie di piani caratterizzanti la singola cristallinità di ogni campione.

La figura \ref{fig: plots} mette in risalto la qualità non ottimale dell'esito delle rilevazioni XRD dei campioni III e VI, probabilmente dovuta a una falla nell'acquisizione dei dati o a una crescita del film non ideale.

Tirando le somme, la riuscita dell'esperienza lascia spazio soltanto al desiderio di estrarre diffrattogrammi più accurati, ovvero con un numero di conteggi medio più elevato, per i casi di studio che hanno risentito di un rapporto segnale rumore al limite dell'accettabilità.


\onecolumn
\newpage
\printbibliography

\newpage
\section*{Appendice}
\begin{figure}[htbp]
    \centering
    \includegraphics[width = 1\linewidth]{../plot_allofit.pdf}
    \caption{Diffrattogrammi a raggi X, acquisiti secondo geometria Bragg-Brentano, di film sottili di zaffiro, in fase di singolo cristallo; necessitano di particolare attenzione i campioni IV e VI che condividono la stessa orientazione N.}
    \label{fig: plots raw}
\end{figure}



\end{document}